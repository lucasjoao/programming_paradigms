\documentclass[12pt]{article}
\usepackage[utf8]{inputenc} % default from sharelatex
\usepackage{indentfirst} % to indent fist paragraph
\usepackage[brazilian]{babel} % BR
\usepackage{setspace} % space between lines

\title{Relatório 01: \\ Panorama histórico}
\author{Lucas João Martins}
\date{}

\begin{document}

\maketitle

\section*{}

\begin{enumerate}
\item \textit{Pesquise sobre o termo ``lógica combinatória'' (ou combinatory logic para info adicional) e tente compreender os combinadores SKI. Consegues
diferenciar os combinadores de operadores?} Lógica combinatória é uma notação introduzida por Moses Schönfinkel e Haskell Curry para eliminar a necessidade de variáveis quantificadas em lógica matemática. Os combinadores SKI são um modelo computacional que podem ser considerados como uma versão reduzida do cálculo lambda não tipado. Todas as operações em cálculo lambda são expressas em SKI como árvores binárias, cujas folhas são um dos três símbolos S, K e I (chamados de combinadores).

\item \textit{O que é a tese de Church-Turing? O que é a prova da computabilidade e como foi construída?} A tese de Church-Turing trouxe uma definição formal para algoritmo e informalmente diz que toda `função que seria naturalmente considerada computável' pode ser computada por uma Máquina de Turing.

\item \textit{Faça uma pesquisa sobre as linguagens de programação existentes e os seus paradigmas.} Há diversos paradigmas, como o lógico, orientado a objeto e o procedural. O lógico é baseado em lógica formal e cada sentença é construída a partir de fatos e regras de um problema (e.g. PROLOG). O orientado a objeto é baseado no conceito de `objetos', onde cada estrutura corresponde a um objeto (e.g. Java). Enquanto que o procedural é composto por rotinas que definem uma sequência de passos computacionais a se seguir (e.g. C).

\item \textit{Lei de Moore, a palestra de Feynman, Peter Shor e a fatoração de
números inteiros grandes.} Lei de Moore foi uma observação proposta por Gordon Moore, da Intel, que dizia que a capacidade de processamento dos computadores em geral dobraria a cada dezoito a vinte e quatro meses. Já Richard Feynman, em 1959, palestrou sobre o controle da manipulação da matéria em escala atômica, dando a ideia da possibilidade da nanotecnologia e computação quântica. Por fim, Peter Shor é o responsável pelo algoritmo de Shor, um algoritmo capaz de fatorar inteiros exponencialmente mais rápido em um computador quântico do que em um computador clássico.

\item \textit{O que é um qubit?} Um qubit é um bit quântico. Ele pode ter três estados: zero, um e intermediário, que trata-se de uma sobreposição de 0 e 1.

\item \textit{D-WAVE?} D-Wave Systems é uma empresa canadense envolvida com computação quântica, que inclusive fabrica computadores quânticos. É considerada a primeira empresa do mundo a vender computadores quânticos. Em 2011 a empresa lançou o D-Wave One, descrito como ``o primeiro computador quântico comercial'', que opera em chips de 128 qubits. Em 2017 ela lançou o o 2000Q, que por sua vez possui 2000 qubits.

\end{enumerate}
\end{document}
